\documentclass{tufte-handout}

%\geometry{showframe}% for debugging purposes -- displays the margins

\usepackage{amsmath}

% Set up the images/graphics package
\usepackage{graphicx}
\setkeys{Gin}{width=\linewidth,totalheight=\textheight,keepaspectratio}
\graphicspath{{graphics/}}

\title{PHOTOVOICE: Moment of hope and fear}
\author[Ale Sanchez-Rios]{Ale Sanchez-Rios}
% \date{}  % if the \date{} command is left out, the current date will be used

% The following package makes prettier tables.  We're all about the bling!
\usepackage{booktabs}

% The units package provides nice, non-stacked fractions and better spacing
% for units.
\usepackage{units}

% The fancyvrb package lets us customize the formatting of verbatim
% environments.  We use a slightly smaller font.
\usepackage{fancyvrb}
\fvset{fontsize=\normalsize}

% Small sections of multiple columns
\usepackage{multicol}

% Provides paragraphs of dummy text
\usepackage{lipsum}

% These commands are used to pretty-print LaTeX commands
\newcommand{\doccmd}[1]{\texttt{\textbackslash#1}}% command name -- adds backslash automatically
\newcommand{\docopt}[1]{\ensuremath{\langle}\textrm{\textit{#1}}\ensuremath{\rangle}}% optional command argument
\newcommand{\docarg}[1]{\textrm{\textit{#1}}}% (required) command argument
\newenvironment{docspec}{\begin{quote}\noindent}{\end{quote}}% command specification environment
\newcommand{\docenv}[1]{\textsf{#1}}% environment name
\newcommand{\docpkg}[1]{\texttt{#1}}% package name
\newcommand{\doccls}[1]{\texttt{#1}}% document class name
\newcommand{\docclsopt}[1]{\texttt{#1}}% document class option name

\begin{document}

\maketitle% this prints the handout title, author, and date

\begin{abstract}
\noindent Description of a picture taken aboard the R/V Sally Ride, one of the few research vessels named after a women. In the picture an undergraduate and a technician are getting ready to deployed an instrument. 
\end{abstract}

%\printclassoptions

% The Tufte-\LaTeX\ document classes define a style similar to the
% style Edward Tufte uses in his books and handouts.  Tufte's style is known
% for its extensive use of sidenotes, tight integration of graphics with
% text, and well-set typography.  This document aims to be at once a
% demonstration of the features of the Tufte-\LaTeX\ document classes
% and a style guide to their use.

\section{Different forces that influence fertility experiences}\label{sec:page-layout}
% \subsection{Headings}\label{sec:headings}
After two weeks together in a cruise, I got to know her better. She had just graduated from college and was eager to study physical oceanography. She was born in India, near the mountains. She missed her family so much. It reminded me of my own privilege of being able to see my family often. I'm a US citizen, so I don't have to worry about visa, and I just need to fly about 2 hours south to get to them. I have been 6 years in my Ph.D. program. Six years, and it's been hard to see any progress or "growth." During this field cruise, I realized how much I have changed from when I started 6 years ago. I was more like this young woman in the picture at the edge of the boat, trusting blindly what people around said, with the idea that they knew more than me. I was so excited that this opportunity was given to me, so grateful that they let me go. During this cruise, she worked for 3 weeks straight, doing 14-hour shifts. She did not receive any compensation, except for the flight to and from the port, nor did she get worker health insurance cover. When I took this picture I was smiling and having fun. She was so excited too. This is a beautiful career, full of friendships and adventures. However this moment also made me sad, this beautiful women of color will realize that this world is not as inviting as it looks like from afar. What has been the prize for me to be able to study this? What have I sacrificed to be here taking this picture? What will she have to? I cannot find the words to warn her, and I do not wish to discourage her either. However I find hard not to share with her the reality of this field. The impact this graduate degree has had in my life I would have never been able to predict it. The emotional turmoil and the internalize self-loathing I have developed was invisible for me the first 4 years of my graduate program. The image I had of myself has deteriorated with time and I’m working hard in building it up. One of the biggest troubles in oceanography is the way they recruit new students, just like the one in this picture, faculty participate in outreach projects in “minority” communities, encouraging students from underrepresented groups to come and study, selecting the “best” of their community and hand picking them the one that best fit their description of excellence. So when you are selected,you feel you won a prize, that you own the college and the faculty something for the favor or letting you in the system. Once you are in the program they assume you will assimilate to the prominent culture, white male insensitive and objective. With the use of microagresssion they try to make clear with characteriztics are not desireble in this field: ‘You have to grow a thicker skin”,” you are to sensitive for science”, “ you need to work on you pronunciation so you are taken seriously”, “ You are too pretty to be working on boats”, “ you have to be more objective”, “Don’t throw away thousands of dollars in you education by becoming a mom”, “ men are just better in math” among many many more. If you are a women of color, transnational, the microagreassion dig even deepr, they mock the way you are used to address superiors, your colleges talk of you country as inferior, and this comments make you feel second best. If a student is unaware all the background that enables others to hurt and to think themselve superior, the system tha empower students with specific traits, as masculine dominant aggresive and egocentristics, they can start thinking they are not smart enough, not good enough, start in a journey of depression and even maybe drop out, which then in the eyes of the progra, confirmed there theory that “ They didn’t had it in them to finish”. I dont want to continue being part of a system that perpetuates the admiration of a dominant culture, however I do not want to walk away from a carrar and field that I have learned to love.  In the three weeks that I have share time with this undergraduate student in the cruise, I saw her wait for validation from my advisor, a white male from Arkansas, a validation that will not come, because he does not know how to give it. I saw her take into heart every comment every critique him or others gave to her, just because they didn't gave her the time to listen to her fully. I saw her questioning my own instructions and asking for validation of other men, a bias I have also acted upon. She looked for me for comfort, but not for knowledge. In a way, she had a feeling I had no power, and we always look for validation from the powerful. She was deploying a microstructure instrument, worth around 100,000, at the edge of the boat, trusting that we all knew what we were doing, trusting that we will support her and guide her and that as long as she works hard it would be ok. A myth that I’m still trying to figure out if is true or not. 

% let's start a new thought -- a new section
% \newthought{In his later books},
% \cite{Tufte2006} 



% \footnote{This is a sidenote that was entered
% using the \texttt{\textbackslash footnote} command.} 

%you can use the %\Verb|\marginnote| command.\marginnote{This is a
%margin note.  Notice that there isn't a number preceding the note, and
%there is no number in the main text where this note was written.}


\section{Feminist Research and Social Justice}\label{sec:installation}

To my understanding, Feminist Research aims to brings different perspectives in to research, and listens as much as it aims to explain. Feminist Research uses historical and social context to challenge the \textit{status quo} of how research is done and understand how can it impact the its results, and consequently impact people lives. Social justice is about trying to understand the underlying connection between communities, their reality and where they are in the spectrum of power dynamics define by how far away one is from being the `norm' ( white, male, heterosexula, christina, able, rich, etc) and Feminist research helps promote research that shed light to this issues and try to make people give it value. It was challenging for me to understand how I can bring feminist methods into my own research. However, I started to understand how the fact that is hard to even envision how that would look like in the field of oceanography, is telling of how embedded are we in an oppressive system that has label this field `objective', removed from human interactions and “above” interpretations. With no room to even evaluate how our own bias and life experiences could affect how we see our results and who do we listen when we collaborate or with whom we work.  There was a very good paper about field glaciology\cite{Carey2016GlaciersScience} and feminist research that started opening my eyes to this field. They talk about the importance of honoring the relationship between ice and communities that have live from the ice for centuries, and how science usually does not give room for this voices. They mentioned how a Feministh approach can help close the gap between this field and communities. 	 
\bibliography{mendeley.bib}
\bibliographystyle{plainnat}
\end{document}
