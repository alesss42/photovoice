\documentclass{tufte-handout}

%\geometry{showframe}% for debugging purposes -- displays the margins

\usepackage{amsmath}

% Set up the images/graphics package
\usepackage{graphicx}
\setkeys{Gin}{width=\linewidth,totalheight=\textheight,keepaspectratio}
\graphicspath{{graphics/}}

\title{"Matters of Choice", by Iris Lopez\cite{Lopez2008MattersFreedom}}
\author[Ale Sanchez-Rios]{Ale Sanchez-Rios}
% \date{}  % if the \date{} command is left out, the current date will be used

% The following package makes prettier tables.  We're all about the bling!
\usepackage{booktabs}

% The units package provides nice, non-stacked fractions and better spacing
% for units.
\usepackage{units}

% The fancyvrb package lets us customize the formatting of verbatim
% environments.  We use a slightly smaller font.
\usepackage{fancyvrb}
\fvset{fontsize=\normalsize}

% Small sections of multiple columns
\usepackage{multicol}

% Provides paragraphs of dummy text
\usepackage{lipsum}

% These commands are used to pretty-print LaTeX commands
\newcommand{\doccmd}[1]{\texttt{\textbackslash#1}}% command name -- adds backslash automatically
\newcommand{\docopt}[1]{\ensuremath{\langle}\textrm{\textit{#1}}\ensuremath{\rangle}}% optional command argument
\newcommand{\docarg}[1]{\textrm{\textit{#1}}}% (required) command argument
\newenvironment{docspec}{\begin{quote}\noindent}{\end{quote}}% command specification environment
\newcommand{\docenv}[1]{\textsf{#1}}% environment name
\newcommand{\docpkg}[1]{\texttt{#1}}% package name
\newcommand{\doccls}[1]{\texttt{#1}}% document class name
\newcommand{\docclsopt}[1]{\texttt{#1}}% document class option name

\begin{document}

\maketitle% this prints the handout title, author, and date

\begin{abstract}
\noindent Case study of Puerto Rican women's fertility experiences.
\end{abstract}

%\printclassoptions

% The Tufte-\LaTeX\ document classes define a style similar to the
% style Edward Tufte uses in his books and handouts.  Tufte's style is known
% for its extensive use of sidenotes, tight integration of graphics with
% text, and well-set typography.  This document aims to be at once a
% demonstration of the features of the Tufte-\LaTeX\ document classes
% and a style guide to their use.

\section{Different forces that influence fertility experiences}\label{sec:page-layout}
% \subsection{Headings}\label{sec:headings}
Using an integral model that looks into the historical, cultural, personal and social level of each individual, we can discuss what the experiences and choices of each of these women. Historically, in the four examples given, was a default option for sterilization as a birth control, which I assume is due to the history of this procedure in Puerto Rico, the fact that sterilization was so common was surprising for me \footnote{Recalling from my own history, it was never mention to me as an option, neither from my family or society. I remember sterilization having a negative connotation as being something not a "good woman" would ever want}. In a personal level, each women had a different life story of how and when sterilization was pursue. They had different ideas of how big they wanted their family to be, but three of them mentioned that they already had at least one more chiled that what they intended to have. It was interesting how Nancy saw sterilization as a way to RESIST sexism, a way to get control over her life that was probably affected by a partner that was not supportive.  Culturally this seems to be also very important, if sterilization is view as women trying to break free from the role the church and society has try to impose on her. Sonia, faces something a little different, she does consider sterilization, but does not want it immediently after the birth of her child. The fact that doctors and nurses tried to convince her, exemplifies the constant fighting women had to do to keep their wishes and have control of their lives, from their own partners and doctors. Though sometimes, because a difference in power this does not occurred, like when the doctors performed a hysterectomy on Carmen. The fact that neither of these women mentioned access to other reversible birth control informs us about their social-economic status. Either because there was no information for them and no economic resources to maintain a source of reversable birth control, this pushed them to resort to La operacion, not as victims without a voice, but as someone presented with hard choices. Social justice help us see in context how, because there is a social agenda from the dominant culture to promote eugenics and to stereotype women of color as hyper-fertile, it so easy for this stories to be true. I mean, there is a reason why doctors are ready to perform this operations without speaking of other options, there is reason why this women have no economic resources to have other birth control and that they have to use this as a way to resist, and is because systematic and institutional oppression. 


% let's start a new thought -- a new section
% \newthought{In his later books},
% \cite{Tufte2006} 



% \footnote{This is a sidenote that was entered
% using the \texttt{\textbackslash footnote} command.} 

%you can use the %\Verb|\marginnote| command.\marginnote{This is a
%margin note.  Notice that there isn't a number preceding the note, and
%there is no number in the main text where this note was written.}


\section{Feminist Research and Social Justice}\label{sec:installation}

To my understanding, Feminist Research aims to brings different perspectives in to research, and listens as much as it aims to explain. Feminist Research uses historical and social context to challenge the \textit{status quo} of how research is done and understand how can it impact the its results, and consequently impact people lives. Social justice is about trying to understand the underlying connection between communities, their reality and where they are in the spectrum of power dynamics define by how far away one is from being the `norm' ( white, male, heterosexula, christina, able, rich, etc) and Feminist research helps promote research that shed light to this issues and try to make people give it value. It was challenging for me to understand how I can bring feminist methods into my own research. However, I started to understand how the fact that is hard to even envision how that would look like in the field of oceanography, is telling of how embedded are we in an oppressive system that has label this field `objective', removed from human interactions and “above” interpretations. With no room to even evaluate how our own bias and life experiences could affect how we see our results and who do we listen when we collaborate or with whom we work.  There was a very good paper about field glaciology\cite{Carey2016GlaciersScience} and feminist research that started opening my eyes to this field. They talk about the importance of honoring the relationship between ice and communities that have live from the ice for centuries, and how science usually does not give room for this voices. They mentioned how a Feministh approach can help close the gap between this field and communities. 	 
\bibliography{mendeley.bib}
\bibliographystyle{plainnat}
\end{document}
